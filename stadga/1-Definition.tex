\section{Definition}

\subsection{Benämning}

\subsubsection{Föreningens fullständiga namn}
Deadly Random Association Within IT

\subsubsection{Föreningens akronym}
DrawIT



\subsection{Föreningens säte}
Göteborg, Västra Götalands län.



\subsection{Föreningsform} \label{sec:föreningsform}
Föreningen är ideell, samt religiöst och partipolitiskt obunden.



\subsection{Syfte} \label{sec:syfte}

\subsubsection{Främjande av intresse}
Föreningen har som syfte att främja intresset för analoga (icke-digitala) spel.



\subsection{Rättigheter}

\subsubsection{Sektionens varumärke}
Föreningen äger rätt att i namn och emblem använda sektionens namn och symboler.



\subsection{Skyldigheter}

\subsubsection{Sektionen}
Föreningen är skyldig att rätta sig efter sektionens stadga, reglemente och fattade beslut.

\subsubsection{Styrelse}
Föreningen måste ha en tillsatt styrelse, se §\ref{sec:sammansättning-styrelse}.

\subsubsection{Årsmöten}
Föreningen måste ha minst ett möte per år dit föreningens och sektionens medlemmar är kallade.



\subsection{Verksamhet}

\subsubsection{Ekonomi}
Föreningens ekonomi skall vara fristående från sektionen.

\subsubsection{Revision}
Föreningens verksamhet och ekonomi kan komma att granskas av sektionens revisorer.



\subsection{Anslutning till Sverok}

\subsubsection{Giltighet}
Paragraf §\ref{sec:anslut-sverok} gäller endast om Sverok betraktar DrawIT som en förening ansluten
till Sverok.

\subsubsection{Anslutning} \label{sec:anslut-sverok}
DrawIT är en förening ansluten till Sverok.
